%\begin{conclusion}
\chapter{Conclusions}
\label{sec_GENcon}

In this thesis the role of preferential sampling has been systematically addressed for the task of geological facies recovery using multiple-point simulation (\emph{MPS}) and for the problem of short-term planning in mining.  In the context of facies recovery using simulations, the task of optimal sampling is formalized and addressed using a maximum information extraction criterion. This sampling principle has the ability to locate samples adaptively on the positions that extract maximum information for the objective of resolving the underlying field. A formal justification is provided in this thesis for adopting this information-driven sampling criterion as well as concrete ways of implementing this principle in practice. In addition, the practical benefits for \emph{MPS} in the context of simulating channelized facies models is demonstrated using synthetic data and real geological facies. Importantly, this strategy locates samples adaptively on the transition between facies which improves the performance of conventional \emph{MPS} algorithms. In conclusion, this work shows that preferential sampling can contribute in \emph{MPS} even at very small sampling regimes and, as a corollary, demonstrates that prior models (obtained form a training image) can be used effectively not only to simulate non-sensed variables of the field, but to decide where to measure next.

Furthermore, the proposed sampling strategy has been adapted to the problem of short-term planning for the task of classifying blocks to be processed as waste or ore in the production stage of a mining project. The problem has been formalized using the principle of  maximum information extraction criterion and the obtained solutions was validated using three data sets of real mining projects.  Importantly, the proposed methodology takes advantage of the information available from the previously sampled locations, allowing to improve the performance as compared with some of the classical non-adaptive sampling schemes used for advanced drilling tasks. From the results obtained across these three real scenarios explored in this thesis, it is possible to see that the proposed methodology achieves better performances than sampling in a structured regular grid (used as a conventional rule for sampling) in terms of both error in image reconstruction and global economic value, when considering the economic revenue of processing the ore and dumping the waste. 

It is important to emphasize that no previous work have addressed the optimal sensing problem covered in this thesis for characterization of geological fields in the context of \emph{MPS}. Overall the main contribution of this thesis relies on the formulation of the sensing design problem and, as a result of that,  the adoption of an adaptive sensing strategy for the characterization of a geological phenomenon. One of the salient aspect of this solution is that the proposed sampling strategy actively uses  \emph{MPS} simulations for  the estimation of the field statistics. At a higher level, this work confirms the fact that \emph{MPS} simulations can be used to estimate empirical representations of the statistics of regionalized variables for better inference and decision.

\section{Future Work}

Although the presented work was focused on 2-D binary channelized structures (geological facies), the applied principles are general and it can be extended to the characterization and recovery of other geological signals with spatial structure in under sampling contexts. There are many directions where this idea could be applied and it is an interesting direction of future research to explore the full potential of this framework. On the specifics, it would be interesting to apply the proposed strategy to scenarios with multiple categories and to use techniques for geostatistical continuous simulation to extend the proposed methodology to continuous variables. Another direction of future work is to study alternative geostatistical simulation tools that could provide more effective estimations of the multi-point patterns (for example using direct sampling). In this direction, the option of estimating the statistics of the model directly from the training image (performing a refined pattern search instead of simulating data) is a very promising.

Finally, although the developed concepts, ideas and algorithms have been developed for inverse problems in geostatistics, the results are applicable to a wide range of disciplines where similar sampling problems need to be faced, included but not limited to design of communication networks, optimal integration and communication of swarms of robots and drones, remote sensing.

Part of the material developed in this thesis has been left out of the main document for reasons of space. Many of these results were significant and therefore the final appendix of this document is dedicated to cover some of these results.



%\end{conclusion}
