Las herramientas \emph{Geoestadísticas} se han convertido en el estándar para la caracterización de la distribución espacial de estructuras geológicas subterráneas. Sin embargo, el problema de recuperación de imágenes en regímenes de bajas tasas de adquisición continúa siendo una tarea desafiante. En la última década, se han desarrollado diversos métodos alternativos para el diseño experimental en estos regímenes, proveyendo nuevas perspectivas para alcanzar mejoras en el desempeño en tareas de reconstrucción y caracterización de imágenes geológicas. Con base en estos logros, un nuevo desafío recae en incorporar herramientas del \emph{estado del arte} en procesamiento de señales y modelamiento estocástico para mejorar este tipo de problemas de inferencia. Esta tesis ha propuesto un estudio profundo de problemas inversos a bajas tasas de muestro con un fuerte enfoque en Geociencias y, en particular, en la reconstrucción de canales binarios de permeabilidad y tareas de control de ley en el contexto de planificación minera de corto plazo.

En este trabajo, la formulación y análisis experimental del problema de \emph{Posicionamiento Optimo de Sensores} (\emph{OSP}) han sido investigados en el contexto de modelos categóricos \emph{2-D} con dependencia espacial. En mineria, este problema intenta encontrar la mejor forma de distribuir las mediciones para optimizar el muestreo/localización de recursos en áreas de exploración y explotación minera. Este trabajo ha apuntado a la formalización del problema {\emph{OSP}} para un numero dado de mediciones disponibles. La caracterización de la incertidumbre es una pieza central de esta formalización. En particular, \emph{OSP} ha sido abordado desde la perspectiva de minimizar la incertidumbre remanente y mediante la implementación de algoritmos secuenciales capaces de estimar dicha incertidumbre. 
El uso de conceptos de teoría de información (\emph{IT}) como \emph{entropía condicional} ha sido considerado para caracterizar la incertidumbre relacionada con modelos geológicos condicionados a la adquisición de datos, y su aplicación en una estrategia de muestreo preferencial orientada a mejorar la inferencia geoestadística en regímenes de bajas tasas de adquisición. La conjetura ha sido que las locaciones basadas en \emph{IT}-\emph{OSP} se distribuyen en la zona de transición de campos aleatorios categóricos, mejorando el desempeño en tareas de recuperación de imágenes comparado con esquemas de muestro no adaptivos clásicos.

A nivel experimental, un algoritmo secuencial regularizado ha sido implementado para aproximar el muestro \emph{IT}-\emph{OSP} para mostrar las ventajas de este principio. El enfoque propuesto proporciona realizaciones basadas en simulaciones multipunto con variabilidad reducida para modelos de facies geológicos categóricos en regímenes de muestreo críticos. Finalmente, el desempeño de los procesos de inferencia bajo las estrategias de muestreo propuestas ha sido evaluado en escenarios prácticos realistas para las tareas de control de ley en el contexto de planificación de corto plazo.
