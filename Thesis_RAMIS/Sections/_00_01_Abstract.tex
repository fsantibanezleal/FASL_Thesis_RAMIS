\emph{Geostatistical} tools have become the standard for characterizing the spatial distribution of geological subsurface structures. However, the problem of image recovery for regimes with low acquisition rates still poses a complex issue. In the last decade, several alternative methods for experimental design at low sampling rates has been developed providing insights into the use of additional prior information to achieve better performance in the reconstruction and characterization of geological images. Based on these achievements, a new challenge is to incorporate tools from the \emph{state of art} in signal processing and stochastic modeling to improve this kind of inference problems. This thesis proposed a comprehensive study of inverse problems at low sampling rates with strong focus on Geosciences and, in particular, for the reconstruction of binary permeability channels and for grade control tasks in short term planning.

%Based on these achievements, the challenge is to incorporate several tools from the \emph{state of art} in signal processing/reconstruction and stochastic modeling to improve this kind of inference problems with special interest in geoscientific applications. We propose a comprehensive study of inference problems at low-rate sampling with strong focus in Geosciences (binary permeability channels).

In this work, the formulation and experimental analysis of the \emph{Optimal Sensor Placement} (\emph{OSP}) problem has been investigated in the context of categorical \emph{2-D} models with spatial dependence. In the mining exploration and production area, this problem attempts to find the best way of distributing measurements (or samples) to optimize sensing/locating resources in areas of mining and drilling. This work aims at formalizing the {\emph{OSP}} problem for a given amount of available measurements. The characterization of the uncertainty is a central piece of this formalization. In particular, the \emph{OSP} problem is addressed from the perspective of minimizing the remaining field uncertainty and sequential algorithms are proposed to solve it. 

The use of \emph{information theoretic} (\emph{IT}) concepts such as \emph{conditional entropy} has been studied to characterize the uncertainty related to a geological model conditioned to the acquisition of data (well logs), and its application in a preferential sampling strategy oriented to improve geostatistical inference at low acquisition rates. The conjecture has been that locations based on \emph{IT}-\emph{OSP} are distributed on transition zones of categorical fields, achieving better performance in tasks of image recovery than standard classical non-adaptive sensing schemes.

In the experimental side, a regularized greedy sequential algorithm is proposed to approximate the proposed \emph{IT}-\emph{OSP} sampling to show this principle. The proposed approach provides realizations based on multiple point simulations with reduced variability for geological categorical facies models in the critical low sampling regime.

Finally, the performances of different inference processes under the proposed sampling strategies are evaluated in some practical realistic scenarios for tasks related with grade control in short term planning. 

%For the systematic analysis concerning the use and integration of \emph{MPS} and training images, a -UNIQUE- model of the
%--MOST PERTINENT ACTUAL CASE-- will be investigated, which would shows a --RELEVANT EXPECTED OUTCOME--- those --RESEARCH TOPIC- that influence the --RELEVANT INFLUENCE AREA--. As some of these --RESEARCH TOPIC- are --REVELANT COMMENTARY

% Sgems is used to create appropiate ,,,
% TIPS is developed as a ....

%Furthermore, --RELEVANT TOPOC- in the process of --RELEVANT PROCCES-- will be investigated, where the focus
%will be on a comparison of ----- as opposed to ----